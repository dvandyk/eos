\documentclass[
    paper=a4paper,
    DIV=15,
    BCRO=10mm,
    fontsize=11pt,
    bibliography=totoc
]{scrbook}

%% Packages
\usepackage[style=phys]{biblatex}                % for the bibliography
\usepackage{booktabs}                            % for book-quality tables
\usepackage{braket}                              % for the braket notation
\usepackage{enumitem}                            % for modifications to the list-like environments
\usepackage[acronym,toc]{glossaries}             % for acronyms and glossaries
\usepackage{hyperref}                            % for hyperlinks in the PDF
\usepackage{listings}                            % for source code and commandline listings
\usepackage{multirow}                            % for table cells spanning multiple rows
\usepackage{todonotes}                           % for margin notes on open tasks
\usepackage{slashed}                             % for Feynman slash notation
\usepackage{palatino}

%% Physics Shortcuts
\newcommand{\dd}{\mathrm{d}}
\newcommand{\eps}{\varepsilon}
\newcommand{\GeV}{\ensuremath{\mathrm{GeV}}}
\newcommand{\MSbar}{\ensuremath{\overline{\text{MS}}}}

%% Math Shortcuts
\newcommand{\para}{\parallel}
\newcommand{\nn}{\nonumber}

%% Abbreviations
\newcommand{\ie}{i.e.}
\newcommand{\eg}{e.g.}
\newcommand{\cf}{cf.}

%%
%% Formatting
%%

%% use serif fonts everywhere

%% create commandline environment
\lstnewenvironment{commandline}[1][\normalsize]{%
\lstset{
    language=bash,
    basicstyle=\scriptsize\ttfamily{#1},
    showspaces=false,
    showtabs=false,
    breaklines=true,
    showstringspaces=false,
    breakatwhitespace=true,
    prebreak={\ttfamily\symbol{'134}},
    %postbreak={\raisebox{0ex}[0ex][0ex]{\ensuremath{\color{red}\hookrightarrow\space}}},
    xleftmargin=.025\textwidth
}
}{}

%% read in an example commandline from file
\newcommand{\commandlineexample}[2][\normalsize]{%
\lstinputlisting[
    language=bash,
    basicstyle={\scriptsize\ttfamily{#1}},
    showspaces=false,
    showtabs=false,
    breaklines=true,
    showstringspaces=false,
    breakatwhitespace=true,
    prebreak={\ttfamily\symbol{'134}},
    %postbreak={\raisebox{0ex}[0ex][0ex]{\ensuremath{\color{red}\hookrightarrow\space}}},
    xleftmargin=.025\textwidth
]{#2}
}

%% create source code environment
\lstnewenvironment{sourcecode}[1][\normalsize]{%
\lstset{
    language=c++,
    basicstyle={\scriptsize\ttfamily{#1}},
    showspaces=false,
    showtabs=false,
    breaklines=true,
    showstringspaces=false,
    breakatwhitespace=true,
    prebreak={\ttfamily\small\symbol{'134}},
    %postbreak={\raisebox{0ex}[0ex][0ex]{\ensuremath{\color{red}\hookrightarrow\space}}},
    xleftmargin=.025\textwidth,
    numbers=left,
    numbersep=5pt,
    numberstyle=\color{gray}\tiny
}
}{}

%% refer to elements of the document
\newcommand{\refapp}[1]{appendix~\ref{app:#1}}
\newcommand{\refeq}[1]{eq.~(\ref{eq:#1})}
\newcommand{\refeqs}[2]{eqs.~(\ref{eq:#1})-(\ref{eq:#2})}
\newcommand{\refsec}[1]{section~\ref{sec:#1}}

%% indent description environments by 2\parindent
\setlist[description]{leftmargin=2\parindent,labelindent=2\parindent}

%% list observables
\newenvironment{observables}{%
\renewcommand{\arraystretch}{1.3}
\begin{center}
\begin{tabular}{l c c}
    \toprule
    Observable name & Mathematical Symbol & Reference\\
    \midrule
}{%
    \bottomrule
\end{tabular}
\end{center}
\renewcommand{\arraystretch}{1}
}
\newcommand{\singleobs}[3]{%
    \texttt{#1} & #2 & #3\\
}
\newcommand{\multiobs}[4]{%
    \texttt{#1} & #2 & \multirow{#4}{*}{#3}\\ \cmidrule{1-2}
}
\newcommand{\multiobsnr}[2]{%
    \texttt{#1} & #2 & \\
}

%% consisten typeset for various objects
\newcommand{\class}[1]{\texttt{#1}}
\newcommand{\cpp}[1]{\texttt{#1}}
\newcommand{\client}[1]{\texttt{#1}}
\newcommand{\package}[1]{\textsf{\textbf{#1}}}


%%
%% Glossaries
%%
\makeglossaries

\newacronym{RNG}{RNG}{Random Number Generator}
\newacronym{LCSR}{LCSR}{Light-Cone Sum Rule}
\newacronym{PDF}{PDF}{Probability Density Function}
\newacronym{PMC}{PMC}{Population Monte Carlo}


%%
%% Metadata
%%
\title{EOS --- A HEP\\ Programm for Flavor Physics}
\subtitle{User Manual}
\author{Danny van Dyk \\ Christoph Bobeth \and Frederik Beaujean}

%% Add the bibliography
\addbibresource{references.bib}

\begin{document}

\frontmatter

%%
%% Title page
%%
\begin{titlepage}
\makeatletter
\KOMAoptions{twoside = false}
\begin{center}

%% Title
\rule{\linewidth}{0.5mm}\\[0.4cm]
{\huge \sffamily \bfseries \@title \\[0.4cm] }
{\large \sffamily \bfseries \@subtitle \\[0.4cm] }
\rule{\linewidth}{0.5mm}\\[1.5cm]

% Authors
\begin{minipage}{0.45\textwidth}
\begin{flushleft} \large \sffamily
\begin{tabular}[t]{c}\@author\end{tabular}
\end{flushleft}
\end{minipage}
\begin{minipage}{0.45\textwidth}
\begin{flushright} \large
\sffamily version XXX
\end{flushright}
\end{minipage}

\vfill
{\large \sffamily \today}

\end{center}
\KOMAoptions{twoside = false}
\makeatother
\end{titlepage}

%% Contents
\tableofcontents

%% List of open ToDo items
\listoftodos

\chapter*{Acknowledgments}

\mainmatter

\part{Documentation}

\chapter{Installation}

\section{Installing the Dependencies}

Installing EOS from source will require the following software:
\begin{description}
    \item[\package{g++}] the GNU C++ compiler, in version 4.8.1 or higher,
    \item[\package{autoconf}] the GNU tool for creating configure scripts, in version 2.69 or higher,
    \item[\package{automake}] the GNU tool for creating makefiles, in version 1.14.1 or higher,
    \item[\package{libtool}] the GNU tool for generic library support, in version 2.4.2 or higher.
\end{description}
%
Building and using the core libraries requires in addition the following software:
\begin{description}
    \item[\package{GSL}] the GNU Scientific Library \cite{GSL}, in version 1.16 or higher,
    \item[\package{HDF5}] the Hierarchical Data Format v5 library \cite{HDF5}, in version 1.8.11 or higher,
    \item[\package{Minuit2}] the physics analysis tool for function minimization, in version 5.28.00 or higher.
\end{description}
Except for \package{Minuit2}, we recommend the installation of the above packages using your system's
software management system. For \package{Minuit2}, we recommend to install from source, and to disable
the automatic support for OpenMP. The installation can be done using:
\begin{commandline}
mkdir /tmp/Minuit2
pushd /tmp/Minuit2
wget http://www.cern.ch/mathlibs/sw/5_28_00/Minuit2/Minuit2-5.28.00.tar.gz
tar zxf Minuit2-5.28.00.tar.gz
pushd Minuit2-5.28.00
./configure --prefix=/opt/pkgs/Minuit2-5.28.00 --disable-openmp
make all
sudo make install
popd
popd
rm -R /tmp/Minuit2
\end{commandline}

If you intend to use the \gls{PMC} sampling algorithm, you will need to install
\begin{description}
    \item[libpmc] a free implementation of said algorithm \cite{libpmc}, in version 1.01 or higher,
\end{description}
in addition to the core library dependencies. The \package{libpmc} package will -- in all likelihood -- not
be installable via your system's software management system. In addition, EOS requires some
modifications to libpmc's source code, in order to make it compatible with C++. We suggest
the following commands to install it:
\begin{commandline}
mkdir /tmp/libpmc
pushd /tmp/libpmc
wget http://www2.iap.fr/users/kilbinge/CosmoPMC/pmclib_v1.01.tar.gz
tar zxf pmclib_v1.01.tar.gz
pushd pmclib_v1.01
./waf configure --m64 --prefix=/opt/pkgs/pmclib-1.01
./waf
sudo ./waf install
sudo find /opt/pkgs/pmclib-1.01/include -name "*.h" \
    -exec sed -i \
    -e 's/#include "errorlist\.h"/#include <pmctools\/errorlist.h>/' \
    -e 's/#include "io\.h"/#include <pmctools\/io.h>/' \
    -e 's/#include "mvdens\.h"/#include <pmctools\/mvdens.h>/' \
    -e 's/#include "maths\.h"/#include <pmctools\/maths.h>/' \
    -e 's/#include "maths_base\.h"/#include <pmctools\/maths_base.h>/' \
    {} \;
sudo sed -i \
    -e 's/^double fmin(double/\/\/&/' \
    -e 's/^double fmax(double/\/\/&/' \
    /opt/pkgs/pmclib-1.01/include/pmctools/maths.h
popd
popd
rm -R /tmp/pmclib
\end{commandline}

\section{Installing EOS}

The most recent version of EOS is contained in the public GIT \cite{GIT} repository.
In order to download it for the first time, create a new local clone of said
repository via:
%
\begin{commandline}
git clone \
    -o eos \
    -b master \
    git://project.het.physik.tu-dortmund.de/eos.git \
    eos
\end{commandline}


As a first step, you need to create all the necessary build scripts via:
%
\begin{commandline}
cd eos
./autogen.bash
\end{commandline}
%
Next, you configure the build scripts using:
%
\begin{commandline}
./configure \
    --prefix=/opt/pkgs/eos \
    --with-minuit2=/opt/pkgs/Minuit2-5.28.00 \
    --with-pmc=no
    # --with-pmc=/opt/pkgs/pmclib-1.01
\end{commandline}
%
Note that the last line is optional, and should replace the second to last line only
if you intend to use the \gls{PMC} sampling algorithm. If the \texttt{configure} script
finds any problems with your system, it will complain loudly.\\

After successful configuration, you can build and install EOS using:
%
\begin{commandline}
make all
sudo make install
\end{commandline}
%
In order to be able to use the EOS clients from the command line, you will need to
\begin{commandline}
export PATH+=":/opt/pkgs/eos/bin"
\end{commandline}
to you \texttt{.bashrc} file. In order to build you own programs that use the EOS libraries,
add
\begin{commandline}
CXXFLAGS+=" -I/opt/pkgs/eos/include"
LDFLAGS+=" -L/opt/pkgs/eos/lib"
\end{commandline}
to your makefile.\\

Moreover, we urgently recommend to also run the complete test suite upon installation,
using:
%
\begin{commandline}
make check
\end{commandline}
%
within the source directory. Please contact the authors in the case that any test failures should occur.


%% vim: set sts=4 et :

\chapter{Usage}
\label{ch:usage}

EOS has been authored with two use cases in mind.\\

The first such use case is the evaluation of observables and further theoretical
quantities in the field of flavor physics. EOS aspires to produce such evaluations
in the course of theory estimates of publication quality.

The second use case is the inference of parameters from experimental observations.
For this task, EOS defaults to the Bayesian framework of parameter inference.\\

In the remainder of this chapter, we document the usage of the existing
EOS clients in order to carry out tasks corresponding to the above use cases.
We assume further that only the built-in observables, physics models and experimental
constraints are used.

\section{Evaluating Observables using \client{eos-evaluate}}
\label{sec:usage:eos-evaluate}


Observables can be evaluated using the \client{eos-evaluate} client. It accepts
the following command line arguments:
\begin{itemize}
    \item[] \texttt{--kinematics NAME VALUE}\\[\medskipamount]
        Within the scope of the next observable, declare a new kinematic
        variable with name \texttt{NAME} and numerical value \texttt{VALUE}.

    \item[] \texttt{--range NAME MIN MAX POINTS}\\[\medskipamount]
        Within the scope of the next observable, declare a new kinematic
        variable with name \texttt{NAME}. Subdivide the interval [MIN, MAX]
        in POINTS subintervals, and evaluate the observable at each subinterval
        boundary.\\

        \emph{Note}: More than one \texttt{--range} command can be issued per
        observable, but only one \texttt{--range} command per kinematic variable.

    \item[] \texttt{--observable NAME}\\[\medskipamount]
        Add a new observable with name \texttt{NAME} to the list of observables
        that shall be evaluated. All previously issued \texttt{--kinematics}
        and \texttt{--range} arguments apply, and will be used by the new obervable.
        The kinematics will be reset (i.e., all kinematic variables will be removed)
        in anticipation of the next \texttt{--observable} argument.
\end{itemize}
The output of calls to \client{eos-evaluate} is structured as follows:
\begin{itemize}
    \item The first row names the observable at hand, as well as all active options.
    \item The second row contains column headers in the order:
        \begin{itemize}
            \item kinematics variables,
            \item the upper and lower uncertainty estimates for each individual uncertainty budget,
            \item the total upper and lower uncertainty estimates.
        \end{itemize}
    \item The third row contains the result as described by the above columns. In addition, at
        the end of the row the relative total uncertainties are given in parantheses.
\end{itemize}
The above structure repeats itself for every observable, as well as for each variation point
of the kinematic variables as described by occuring \texttt{--range} arguments.\\

As an example, we turn to the evaluation of the $q^2$-integrated branching ratio
$\mathcal{B}(\bar{B}^0\to \pi^+\ell^-\bar\nu_\ell)$; see \refsec{physics:b-mesons:cc:Btopi-semileptonic}
for the definition.
For this example, let us use the integration range
\begin{equation*}
    0\,\GeV^2 \leq q^2 \leq 12\,\GeV^2\,.
\end{equation*}
Further, let us use the BCL2008 \cite{Bourrely:2008za} parametrization of the $\bar{B}\to \pi$ form factor,
as well as the Wolfenstein parametrization of the CKM matrix. The latter is achieved
by choosing the physics model 'SM'. By default, EOS uses the most recent results of
the UTfit collaboration's fit of the CKM Wolfenstein parameter to data on tree-level decays.
In this example, we will evaluate the observable, and estimate parametric uncertainties
based on the naive expectation of Gaussian uncertainty propagation. We will classify two
budgets of parametric uncertainties: one for uncertainties pertaining to the form factors
(labelled 'FF'), and one for uncertainties pertaining to the CKM matrix elements (labelled 'CKM').

Our intentations translate to the following call to \client{eos-evaluate}:
\commandlineexample{examples/evaluate-btopilnu-integrated}

\section{Finding the Mode of a Probability Density using \textbf{\client{eos-find-mode}}}
\label{sec:usage:eos-find-mode}

The best-fit point, or simply the most-likely value of some \gls{PDF} $P(\vec\theta)$ is
regularly searched for. EOS supplies the client \client{eos-find-mode}, which accepts the
following command-line arguments:
\begin{itemize}
    \item[] \texttt{--scan NAME --prior flat MIN MAX}\\[\medskipamount]

    \item[] \texttt{--scan NAME [ABSMIN ABSMAX] --prior gaussian MIN CENTRAL MAX}\\[\medskipamount]

    \item[] \texttt{--nuisance}\\[\medskipamount]
        The \texttt{--nuisance} argument works identically as the \texttt{--scan} argument,
        with one exception: It declare the associated parameter as a nuisance parameter, which
        is flagged in the HDF5 output. The sampling algorithm \emph{does not} treat nuisance
        parameters differently than scan parameter.
\end{itemize}

As an example, we define the a-priori \gls{PDF} for a study of the decay $\bar{B}\to \pi^+\ell^-\bar\nu_\ell$.
For the CKM Wolfenstein parameters, we use
\begin{equation*}
\begin{aligned}
    \lambda    & = 0.22535 \pm 0.00065\,,  &
    A          & = 0.807 \pm 0.020\,,      \\
    \bar{\rho} & = 0.128 \pm 0.055\,,      &
    \bar{\eta} & = 0.375 \pm 0.060\,.
\end{aligned}
\end{equation*}
For the a-priori \gls{PDF}, we use uniform distributions for the BCL2008 \cite{Bourrely:2008za}
parameters. However, we construct a likelihood from the results of a recent study of the form factor
$f^{B\pi}_+(q^2)$ within \glspl{LCSR}. 


\section{Producing Random Parameter Samples using \textbf{\texttt{eos-sample-mcmc}}}
\label{sec:usage:eos-sample-mcmc}

Both use cases, observable evaluation and Bayesian parameter inference, make use of
random samples of some \gls{PDF} $P(\vec\theta)$. These random
samples can be produced from Markov chains, using the Metropolis-Hastings algorithm,
by calls to the \client{eos-sample-mcmc} client. It accepts the following command-line arguments:
\begin{itemize}
    \item[] \texttt{--scan NAME --prior flat MIN MAX}\\[\medskipamount]

    \item[] \texttt{--scan NAME [ABSMIN ABSMAX] --prior gaussian MIN CENTRAL MAX}\\[\medskipamount]

    \item[] \texttt{--nuisance}\\[\medskipamount]
        The \texttt{--nuisance} argument works identically as the \texttt{--scan} argument,
        with one exception: It declare the associated parameter as a nuisance parameter, which
        is flagged in the HDF5 output. The sampling algorithm \emph{does not} treat nuisance
        parameters differently than scan parameter.

    \item[] \texttt{--seed [time|VALUE]}\\[\medskipamount]
        This argument sets the seed value for the \gls{RNG}. Setting the
        seed to a fixed numerical \texttt{VALUE} ensures reproducibility of the results. This
        is important for publication-quality usage of the client. If \texttt{time} is
        specified, the \gls{RNG} is seeded with an interger value based on the current time.

    \item[] \texttt{--prerun-min VALUE}\\[\medskipamount]
        For the prerun phase of the sampling algorithm, set the minimum number of
        steps to \texttt{VALUE}.

    \item[] \texttt{--prerun-max}\\[\medskipamount]
        For the prerun phase of the sampling algorithm, set the maximum number of
        steps to \texttt{VALUE}.

    \item[] \texttt{--prerun-update}\\[\medskipamount]
        For the prerun phase of the sampling algorithm, force an adaptation of the
        Markov chain's proposal function to its environment after every \texttt{VALUE}
        steps.

    \item[] \texttt{--store-prerun [0|1]}\\[\medskipamount]
        Either disable or enable storing of the prerun samples to the output file.\\

        \emph{Note}: Samples from the prerun should only be used for diagnostic purpose.

    \item[] \texttt{--output FILENAME}\\[\medskipamount]
        Use the file \texttt{FILENAME} to store the output, using the HDF5 file format.
\end{itemize}

We consider the same example as in \refsec{sec:usage:eos-find-mode}. However, now we intend
to draw random numbers from the PDF using EOS' adaptive Metropolis-Hasting algorithm.
During its prerun phase, the algorithm adapts the chains' proposal
functions. We should demand at least $500$, and -- for a problem of this complexity -- maximally
$7500$ steps during the prerun phase; the adaption process should be executed
after every $500$ steps. For the final samples, we wish for a total of $10^4$, which we artifically decompose
into $10$ chunks with $1000$ samples each.\\

Our intentions translate to the following call to \client{eos-sample-mcmc}:
\commandlineexample{examples/sample-mcmc-btopi-ff}


\section{Bayesian Uncertainty Propagation using \textbf{\texttt{eos-propagate-uncertainty}}}
\label{sec:usage:eos-propagate-uncertainty}

Our intentions translate to the following call to \client{eos-sample-mcmc}:\todo{Does not work yet!}
\commandlineexample{examples/propagate-uncertainty-b-to-pi-l-nu}

\section{List of Observables}
\label{sec:usage:observables}

In this section we list of observable and other evaluatable quantities within
EOS.

\begin{table}[h]
\begin{observables}
% B -> pi
    \multiobs{B->pi::f\_+(s)}{$f^{B\pi}_+(q^2)$}{\refeq{physics:hme:BtoP-ff-V} with $B=\bar{B}_q$ and $P=\pi$}{2}
    \multiobsnr{B->pi::f\_0(s)}{$f^{B\pi}_0(q^2)$}\midrule
    \singleobs{B->pi::f\_T(s)}{$f^{B\pi}_T(q^2)$}{\refeq{physics:hme:BtoP-ff-T} with $B=\bar{B}_q$ and $P=\pi$}
    \midrule
% B -> K
    \multiobs{B->K::f\_+(s)}{$f^{BK}_+(q^2)$}{\refeq{physics:hme:BtoP-ff-V} with $B=\bar{B}_q$ and $P=\bar{K}$}{2}
    \multiobsnr{B->K::f\_0(s)}{$f^{BK}_0(q^2)$}\midrule
    \singleobs{B->K::f\_T(s)}{$f^{BK}_T(q^2)$}{\refeq{physics:hme:BtoP-ff-T} with $B=\bar{B}_q$ and $P=\bar{K}$}
\end{observables}
\caption{Form factors}
\end{table}


%% vim: set sts=4 et tw=100 :

\chapter{Library Interface}
\label{ch:interface}

We begin this chapter by explaining the rationale behind several of the design decisions of the
EOS libraries in \refsec{interface:design}. Subsequently, we document the core set of C++ classes
in \refsec{interface:classes}.


\section{Design}
\label{sec:interface:design}

In order to fulfill its intended use cases, the EOS libraries are designed with concepts in mind.\\

First, most of the scalar quantities that used within the EOS libraries are treated as parameters.
These start with directy experimental input, such as particle masses and lifetimes. They continue
along the lines of more theoretically motivated quantities, such as quark masses (in the \MSbar{}
scheme) and the Wolfenstein parameters of the CKM matrix. It is therefore straightforward to change
a parameter's role within the scope of a theoretical analysis, from being a nuisance parameter in
the course of producing some estimates to being a genuine parameter of interest in the course of a
fit. In order to differentiate between the various parameters, a naming scheme is put in place.
Within this scheme, a parameter's name is rendered:
\begin{equation}
    \texttt{NAMESPACE::ID@TAG}\,,
\end{equation}
where the meta variables take the following meaning:
\begin{description}
    \item[NAMESPACE] ... \\
    \item[ID] ...\\
    \item[TAG] ...\\
\end{description}

Second, if a quantity exhibits a functional dependence on either a parameter or a kinematic variable,
it is treated in a modular fashion. Per default, an abstract interface is created that allows for
several implementation of the same quantity. The actual implementations of this interface are then
accesible via a factory method. An excellent example for such a \emph{plugin} design is found within
the scope of hadronic matrix elements, in particular hadronic form factors; see e.g.
\refsec{physics:b-mesons:hme}. Each implementation of a plugin usually depends on its own set of
parameters. For clarity, we use one of the hadronic form factor as an example. Consider the $B\to
\pi$ form factor $f^{B\pi}_+$ as defined in \refeq{physics:hme:BtoP-ff-V}. One possible
parametrization of this form factor has been dubbed BCL2008 \cite{}. It is achieved in terms of
three parameters: the normalization $f^{B\pi}_+(0)$, and two shape parameters $b^{B\pi,+}_1$ and
$b^{B\pi,+}_2$. This parametrization is implemented within EOS, andobservables that depend on the
$B\to \pi$ form factors can choose it through the option \texttt{form-factors=BCL2008}. The relevant
parameters are contained in the namespace \texttt{B->pi}, and tagged for the \text{BCL2008} plugin:
\begin{equation}
    \texttt{B->pi::f\_+(0)\@BCL2008}\,,\quad\texttt{B->pi::b\_+\^1\@BCL2008}\,,
    \quad\text{and}\quad\texttt{B->pi::b\_+\^2\@BCL2008}\,.
\end{equation}
As such, there is no risk of namespace collisions among the various plugins' parameters.\\


Third, \dots

{Likelihood and Prior}
\begin{itemize}
    \item construct likelihood and prior at run time
    \item abstract tree, with leaves:
    \begin{itemize}
        \item (Multivariate) Gaussian distribution
        \item LogGamma distribution (for asymmetric uncertainty intervals)
        \item Amoroso (for limits)
        \item Flat (prior only)
    \end{itemize}
\end{itemize}


\section{Core Classes}
\label{sec:interface:classes}

At the core of the EOS API there are four classes -- \class{Parameters}, \class{Kinematics},
\class{Options}, \class{Observable} -- all of which are discussed in the following.

\subsection{Class \class{Parameters}}

\texttt{key = value} dictionary, with string keys and floating-point real values\\

\begin{itemize}
    \item copies share, by default, the parameters of the original
    \item observables usually share a common set of parameters
\end{itemize}

\begin{figure}[t]
    \centering
    \includegraphics[width=.5\textwidth]{figures/graph-parameters.pdf}
\end{figure}

\begin{sourcecode}
Parameters paramA = Parameters::Defaults();
Parameters paramB(paramA);

ObservablePtr obsA = Observable::make("A", paramA, Kinematics{ }, Options{ });
ObservablePtr obsB = Observable::make("A", paramB, Kinematics{ }, Options{ });
\end{sourcecode}

\begin{itemize}
    \item access to individual parameters via array subscript \texttt{[\,]}
    \begin{itemize}
        \item input: parameter name
        \item result: instance of \class{Parameter},
            w/ persistent access to parameter data\\
            {lookup once, use often!}
    \end{itemize}
    \item parameter naming scheme: \texttt{NAMESPACE::ID@SOURCE}, e.g.:\\
    \begin{itemize}
        \item \texttt{mass::b(MSbar)} $\to$ mass $\overline{m}_b(\overline{m}_b)$ in \MSbar{} scheme
        \item \texttt{B->K::f\_+(0)@KMPW2010} $\to$ normalization of $f_+$ FF in $B\to K$ decays, according to KMPW2010
    \end{itemize}
\end{itemize}

\subsection{Class \class{Kinematics}}

The class \class{Kinematics} is a dictionary from \cpp{std::string}-valued keys to
\cpp{double}-valued entries. Upon construction of an observable, a suitable instance
of kinematics is bound to that observable.

\texttt{key = value} dictionary, with string keys and floating-point real values

\begin{itemize}
    \item allows run-time construction of observables
    \item each obervable has its very own set of kinematic variables
    \item access to individual variables via array subscript {\texttt{[\,]}}
    \begin{itemize}
        \item input: variable name
        \item result: double
    \end{itemize}
    \item no naming scheme, since namespace is unique per observable instance
\end{itemize}

\subsection{Class \class{Options}}

\texttt{key = value} dictionary, with string keys and string values
influences how observables are evaluated

\begin{itemize}
    \item access to individual otions via array subscript {\texttt{[\,]}}
    \begin{itemize}
        \item input: option name
        \item result: string value
    \end{itemize}
    \item example: lepton flavour in semileptonic decay:\\
        \texttt{l=mu}, \texttt{l=tau}, \dots
    \item example: choice of form factors:\\
        \texttt{form-factors=KMPW2010}\, \dots
    \item example: \texttt{model=\dots} as choice of underlying physics model
    \begin{description}
        \item[\texttt{SM}] to produce SM prediction
        \item[\texttt{WilsonScan}] to fit Wilson coefficients
        \item[\texttt{CKMScan}] to fit CKM matrix elements
    \end{description}
\end{itemize}

\subsection{Class \class{Observable}}

\class{Observable} is an abstract base class\\

\begin{itemize}
    \item descendants must at construction time:
        \begin{itemize}
            \item associate with instance of Parameters
            \item extract value from instance of Options
        \end{itemize}

    \item construction via factory method:\\
       create an observable at runtime using its name, a set of parameters, a set of
       kinematic variables, and a set of options:\\[\medskipamount]

       \cpp{Observable::make("B->pilnu::BR", p, k, o)}

    \item changes to \class{Parameters} transparently affect associated observables

    \item changes to \class{Options} do not affect the observable after construction

    \item observables can be
        \begin{itemize}
            \item evaluated:\\[\smallskipamount]
            runs the necessary computations for the present values of the parameters\\[\smallskipamount]
            \item copied:\\[\smallskipamount]
            copy-ctor does not create an independent copy, the copy uses the same
            parameters, with the same options\\[\smallskipamount]

            \item cloned:\\[\smallskipamount]
            creates an independent copy of the same observable, using a different set of
            parameters than the original
        \end{itemize}

    \item all users of \class{Observable} must also support cloning\\
        \begin{itemize}
            \item easily allows to parallelize algorithms
        \end{itemize}
\end{itemize}

\includegraphics[width=.8\textwidth]{figures/graph-observable-copy.pdf}
\includegraphics[width=.8\textwidth]{figures/graph-observable-clone.pdf}


\chapter{Tutorials}

\section{How to add a new observable}

\section{How to add a new measurement}

%%
%% Physics
%%
\part{Physics}

\chapter{Effective Field Theories}


\newcommand{\op}[1]{\mathcal{O}_{#1}}
\newcommand{\wilson}[2][{}]{\mathcal{C}_{#2}^{\mathrm{#1}}}

%--------+---------+---------+---------+---------+---------+---------+---------+
%
%  intro to physics section
%
%--------+---------+---------+---------+---------+---------+---------+---------+

The electro-weak decays of hadrons (mesons and baryons) with masses much smaller
than the electro-weak scale of order of the $W$-boson mass, $m_W$, can be efficiently
described using effective field theories (EFT) in the spirit of the well-known
Fermi theory of the $\beta$-decay. This comprises practically all hadrons containing
light quarks $q = (u,\,d,\,s,\,c,\,b)$. In this context, specific flavour-changing
higher-dimensional ($dim > 4$) operators arise in the standard model (SM) and
it's extensions, accompanied by effective couplings (Wilson coefficients) giving
rise to the generic structure of the effective Lagrangian
\begin{align}
  {\cal L}_{\rm EFT} &
  = {\cal L}_{{\rm QCD} \times {\rm QED}} (\mbox{light particles})
  + \sum_i \wilson{i}(\mu) \op{i} + \mbox{h.c.} + \ldots \,.
\end{align} 
The first term describes the $SU(3)_c \times U(1)_{\rm em}$
gauge interactions of all light quarks, $q$, and leptons, $\ell = (e,\,\mu,\,\tau)$,
with QCD and QED gauge bosons. The second part are the aforementioned operators
$\op{i}$ and effective couplings $\wilson{i}$, where the latter depend on a
factorization scale $\mu_{\rm low}$ that is usually of the order of the mass of the
decaying hadron. The operators are composed out of light degrees of freedom, i.e. fermions
$q$ and $\ell$, as well as $SU(3)_c$ and $U(1)_{\rm em}$ gauge bosons. Beyond
the SM, it is conceivable that in principle additional light degrees of freedom
exist, which however must have escaped direct detection so far.
The Wilson coefficients are suppressed by inverse powers of the electroweak
scale or some new physics scale, depending on the dimension of their operators.
Finally, the dots denote higher-dimensional operators that have been neglected.
In practical applications, assuming the SM, they are suppressed at least by
$m_b^2/m_W^2 \sim 0.004$, where $m_b$ denotes the bottom-quark mass.

The EFT framework is sufficient to describe interactions at and below the scale
$\mu_{\rm low}$, including hadronic binding effects due to strong interactions
as well as electro-magnetic virtual and real corrections to observables. The
implementation of according observables in EOS is thus based on universal EFT 
Lagrangians. All respective short-distance interactions above $\mu_{\rm low}$ are
fixed by the structure of the operators and the values of the Wilson coefficients.
In the spirit of Fermi, the according Wilson coefficients can be viewed as unknowns
to be determined from experiment. This so-called {\em model-independent} approach
implies the independence of all Wilson coefficients and correlation among observables
arise from the assumption of which operators are included, i.e. have non-vanishing
Wilson coefficients at the scale $\mu_{\rm low}$ (or some other). 

The latter point
is important in view of operator mixing (under QCD and QED), which is governed by
the anomalous dimension matrices (ADM) of the involved operators. In the case of
operator mixing, vanishing Wilson coefficients at some scale $\mu_0$ can become 
non-zero at some other scale $\mu_1$, depending on their mixing and the initial
conditions of Wilson coefficients involved in the mixing. This is summarised
by the general solution of the renormalisation group equation (RGE)
\begin{align}
  \wilson{i}(\mu_1) & = \sum_j [U(\mu_1, \mu_0)]_{ij} \wilson{j}(\mu_0)\,.
\end{align}
The evolution matrix $U$ depends on the ADM's of the operators, the strong
and electro-magnetic couplings, $\alpha_s$ and $\alpha_e$ respectively, and
their respective RG-functions (beta-functions). Throughout Wilson coefficients
and couplings are $\overline{\rm MS}$-renormalized quantities.

In the following we collect the conventions of the effective theories implemented
in EOS for hadron decays based on quark transitions
\begin{align}
  b & \to (d,\, s), & \ldots
\end{align}
For an introduction to the topic, the reader is referred to the exhaustive review
articles \cite{Buchalla:1995vs, Buras:1998raa}. 

The following abbreviations will be often used below for products of elements
of the CKM quark mixing matrix appearing in $b$-quark decays
\begin{align}
  \lambda_U^{(D)} & = V_{Ub}^{} V_{UD}^* \,, &
  D & = (d,\,s), \, &
  U & = (u,\,c,\,t) \,.
\end{align}

%--------+---------+---------+---------+---------+---------+---------+---------+
%
%  b -> s  EFT
%
%--------+---------+---------+---------+---------+---------+---------+---------+

\section{$|\Delta B| = |\Delta S| = 1$}

In this section we summarise the convention of the EFT for $|\Delta B| = |\Delta S| = 1$
decays covering transitions
\begin{align*}
  b & \to s + (\bar{u}u,\, \bar{c}c)           && \mbox{current-current} \,,
\\
  b & \to s + \bar{q}q                         && \mbox{QCD \& QED penguin} \,,
\\
  b & \to s + (\gamma,\, \mbox{gluon})         && \mbox{electro- \& chromo-magnetic dipole} \,,
\\
  b & \to s + (\bar{\ell}\ell,\, \bar{\nu}\nu) && \mbox{semi-leptonic} \,,
\end{align*}
where the classification corresponds to the origin of the operators from 
decoupling $W,Z$-bosons and the top-quark in the SM. The basis contains several
blocks that are however related via operator mixing and renders the RGE of the
Wilson coefficients non-trivial. The results for $|\Delta B| = |\Delta D| = 1$
transitions are obtained by the replacement $s\to d$.
 
We start with the operators generated in the SM, where the initial Wilson coefficients
and ADM's are known at several orders in perturbation theory.
The most appropriate choice of basis for higher order QCD calculations of ADM's was
given in \cite{Chetyrkin:1996vx, Chetyrkin:1997gb} with according extension for
QED corrections in \cite{Bobeth:2003at, Huber:2005ig}. The Lagrangian
\begin{equation}
\begin{aligned}
  {\cal L}_{\rm EFT} &
  = {\cal L}_{{\rm QCD} \times {\rm QED}} (q;\, \ell)
  + \frac{4 G_F}{\sqrt{2}} \lambda_{u}^{(s)} \sum_{i=1}^{2} 
     \wilson{i} (\op{i}^c - \op{i}^u)
\\
  & + \frac{4 G_F}{\sqrt{2}} \lambda_{t}^{(s)} \left[
      \sum_{i=1}^{2}  \wilson{i} \op{i}^c
    + \sum_{i=3}^{10} \wilson{i} \op{i}
    + \sum_{i=3}^{6}  \wilson{iQ} \op{iQ}
    + \wilson{b}\op{b} \right] + \mbox{h.c.} \,.
\end{aligned}
\end{equation}
incorporates unitarity of the CKM matrix  $\lambda_u^{(s)} + \lambda_c^{(s)} + 
\lambda_t^{(s)} = 1$. All Wilson coefficients $\wilson{i}$ are evaluated at
$\mu_{\rm low} \sim m_b$. The up-quark
sector $\sim \lambda_u^{(s)}$ is doubly-Cabibbo suppressed for $b\to s$
transitions and leads to tiny CP-asymmetries in the SM.

The current-current ($U = u,\, c$) operators are defined as 
\begin{equation}
\begin{aligned}
  \op{1}^U & = (\bar{s} \gamma_\mu P_L T^a U) (\bar{U} \gamma^\mu P_L T^a b) \,, &
  \op{2}^U & = (\bar{s} \gamma_\mu P_L U) (\bar{U} \gamma^\mu P_L b) \,, 
\end{aligned}
\end{equation}
which arise already at tree-level from decoupling the $W$-boson in $b\to s + 
(\bar{u}u,\, \bar{c}c)$ and mix into all other operators, except the
semi-leptonic $\op{10}$. Here and below $P_{L,R} = (1 \mp \gamma_5)/2$ denote
chirality projectors. 

The QCD-penguin operators are
\begin{equation}
\begin{aligned}
  \op{3} & = (\bar{s} \gamma_\mu P_L b)     \sum_q (\bar{q} \gamma^\mu q) \,, &
  \op{5} & = (\bar{s} \gamma_{\mu\nu\rho} P_L b)
             \sum_q (\bar{q} \gamma^{\mu\nu\rho} q) \,,
\\[1mm] 
  \op{4} & = (\bar{s} \gamma_\mu P_L T^a b) \sum_q (\bar{q} \gamma^\mu T^a q) \,, &
  \op{6} & = (\bar{s} \gamma_{\mu\nu\rho} P_L T^a b) 
             \sum_q (\bar{q} \gamma^{\mu\nu\rho} T^a q) \,,
\end{aligned}
\end{equation}
where the sum extends over all $q = (u,\,d,\,s,\,c,\,b)$ and $T^a$ are the
generators of $SU(3)_c$. Products of several gamma matrices have been
abbreviated $\gamma^{\mu\nu\rho} \equiv \gamma^\mu\gamma^\nu \gamma^\rho$. 

Analogous QED-penguin operators are defined as 
\begin{equation}
\begin{aligned}
  \op{3Q} & = (\bar{s} \gamma_\mu P_L b)     \sum_q Q_q (\bar{q} \gamma^\mu q) \,, & 
  \op{5Q} & = (\bar{s} \gamma_{\mu\nu\rho} P_L b)
             \sum_q Q_q (\bar{q} \gamma^{\mu\nu\rho} q) \,, &
\\[1mm]   
  \op{4Q} & = (\bar{s} \gamma_\mu P_L T^a b) \sum_q Q_q (\bar{q} \gamma^\mu T^a q) \,, &
  \op{6Q} & = (\bar{s} \gamma_{\mu\nu\rho} P_L T^a b) 
             \sum_q Q_q (\bar{q} \gamma^{\mu\nu\rho} T^a q) \,,
\end{aligned}
\end{equation}
where $Q_q$ denotes the quark charges as multiples of $e$. Further, an additional
operator has to be considered
\begin{align}
  \op{b} & = -\frac{1}{3}  (\bar{s} \gamma_\mu P_L b)(\bar{b} \gamma^\mu b)
             +\frac{1}{12} (\bar{s} \gamma_{\mu\nu\rho} P_L b)
                           (\bar{b} \gamma^{\mu\nu\rho} b) \,,
\end{align}
receiving contributions from electro-weak boxes. In four dimension it would
correspond to $(\bar{s} \gamma_\mu P_L b)(\bar{b} \gamma^\mu P_L b)$, however
the above form allows to avoid traces with $\gamma_5$ to all orders in QCD.

The electro- and chromo-magnetic dipole operators
\begin{align}
  \op{7} & = \frac{e}{g_s^2} [\bar{s} \sigma^{\mu \nu} 
             (\overline{m}_b P_R + \overline{m}_s P_L) b] F_{\mu \nu} \,, &
  \op{8} & = \frac{1}{g_s}   [\bar{s} \sigma^{\mu \nu}
             (\overline{m}_b P_R + \overline{m}_s P_L) T^a b] G_{\mu \nu}^a \,, &  
\end{align}
receive contributions from on-shell photon and gluon penguins. The appearing
quark masses are renormalised in the $\overline{\mbox{MS}}$ scheme. 

In the SM there are two semi-leptonic operators
\begin{align}
  \op{9}  & = \frac{e^2}{g_s^2} (\bar{s} \gamma_\mu P_L b) 
              \sum_\ell (\bar{\ell} \gamma^\mu \ell) \,, &
  \op{10} & = \frac{e^2}{g_s^2} (\bar{s} \gamma_\mu P_L b)
              \sum_\ell (\bar{\ell} \gamma^\mu \gamma_5 \ell) \,, &  
\end{align}
describing $b\to s + \bar\ell\ell$ transitions. In this case the lepton
charge $Q_\ell$ has been pulled into the definition of the Wilson
coefficient.

The normalization to $4\pi/g_s^2$, the QCD coupling, has been chosen for practical
reasons such that the leading SM 1-loop correction to the initial Wilson coefficients
counts formally as a strong correction rather then an electro-magnetic one. The 
initial Wilson coefficients are known up to NNLO in QCD and NLO in EW corrections
\begin{align}
  \wilson{i}(\mu_0) = 
\end{align}
($a_i \equiv \alpha_i/(4\pi)$)

{\em note definition of evanescent operators, without whom ADM's and initial Wilson 
coefficients are meaningless ...} 




\chapter{$b$-Meson decays}

\section{Hadronic Matrix Elements}
\label{sec:physics:b-mesons:hme}

\subsection{$B\to P$ Form Factors}
\label{sec:physics:b-mesons:hme:BtoP}

The hadronic matrix elements for $B\to P$ transitions involving
the vector current, where $P$ denotes a leight pseudoscalar meson), are parametrized as
\begin{equation}
\label{eq:physics:hme:BtoP-ff-V}
    \bra{P(k)} \bar{q} \gamma_\mu b \ket{\bar{B}(p)}
    = f^{BP}_+(q^2) (p + k)^\mu + \left[f^{BP}_0(q^2) - f^{BP}_+(q^2)\right] \frac{M_B^2 - M_P^2}{q^2} q^\mu\,.
\end{equation}
Here $q^\mu \equiv (p - k)^\mu$ represent the momentum transfer. For the tensor current
we parametrize similarly
\begin{equation}
\label{eq:physics:hme:BtoP-ff-T}
    \bra{P(k)} \bar{q} \sigma^{\mu\nu} b \ket{\bar{B}(p)}
    = \frac{i f^{BP}_T(q^2)}{M_B + M_P} \left[(p + k)^\mu q^\nu - q^\mu (p + k)^\nu\right]\,.
\end{equation}

\subsection{$B\to V$ Form Factors}
\label{physics:b-mesons:hme:BtoV}

The hadronic matrix elements for $B\to V$ transitions involving
the vector and axialvector current, where $V$ denotes a leight pseudoscalar meson), are parametrized as
\begin{equation}
\label{eq:physics:hme:BtoV-ff-V}
    \bra{V(k, \eta)} \bar{q} \gamma_\mu b \ket{\bar{B}(p)}
    = \dots
\end{equation}
and
\begin{equation}
\label{eq:physics:hme:BtoV-ff-A}
    \bra{V(k, \eta)} \bar{q} \gamma_\mu \gamma_5 b \ket{\bar{B}(p)}
    = \dots\,.
\end{equation}
Here $q^\mu \equiv (p - k)^\mu$ represent the momentum transfer. The pseusoscalar current involves
\begin{equation}
\label{eq:physics:hme:BtoV-ff-P}
    \bra{V(k, \eta)} \bar{q} \gamma_5 b \ket{\bar{B}(p)}
    = \dots\,,
\end{equation}
while the matrix elements of the scalar current vanish.
For the tensor current we parametrize similarly
\begin{equation}
\label{eq:physics:hme:BtoV-ff-T}
    \bra{V(k, \eta)} \bar{q} \sigma^{\mu\nu} b \ket{\bar{B}(p)}
    = \dots\,.
\end{equation}
Note that the above depends on the sign convention of the Levi-Civita tensor, see \refapp{conventions}.


\section{Semileptonic $B$ decays from Charged Currents}

%% vim: set sts=4 et :
\subsection{The decay $\bar{B}^0\to \pi^+ \ell^- \bar\nu_\ell$ (for $\ell = e,\mu$}
\label{sec:physics:b-mesons:cc:Btopi-semileptonic}

The differential decay width of the decay $\bar{B}^0\to \pi^+\ell^-\bar\nu_\ell$ reads \cite{Bourrely:2008za}
\begin{equation}
    \frac{\dd \Gamma}{\dd q^2}
\end{equation}
where $q^2$ corresponds to the momentum transfer to the lepton pair.


\section{Rare radiative and semileptonic $B$ decays from Flavor Changing Neutral Currents}

\subsection{The inclusive decay $\bar{B}\to X_s \gamma$}
\label{physics:b-mesons:fcnc:incl-b-to-s-dilepton}




\subsection{The inclusive decay $\bar{B}\to X_s \ell^+\ell^-$}
\label{physics:b-mesons:fcnc:incl-b-to-s-dilepton}




%%% Shortcuts
%\newcommand{\modified}[1]{{\textcolor{red}{{#1}}}}
%\newcommand{\todo}[1]{{\textcolor{red}{\text{\textbf{ToDo:}\,{#1}}}}}
%\newcommand{\christoph}[1]{{\textcolor{blue}{(\textbf{Christoph:}\,{#1})}}}
%\newcommand{\danny}[1]{{\textcolor{magenta}{(\textbf{Danny:}\,{#1})}}}
%\newcommand{\citeneeded}{\textsuperscript{\textcolor{red}{Citation needed}}}
%\Renewcommand{\left(}{\left(}
%\Renewcommand{\right)}{\right)}
%%\Renewcommand{\lbrace}{\left\lbrace}
%%\Renewcommand{\rbrace}{\right\rbrace}
%\Renewcommand{\Re}[1]{\mathrm{Re}\left(#1\right)}
%\Renewcommand{\Im}[1]{\mathrm{Im}\left(#1\right)}
%\newcommand{\dd}[1][{}]{\mathrm{d}^{#1}\!\!\;}
%\newcommand{\rmd}{\rm d}
%\newcommand{\del}{\partial}
%\newcommand{\nn}{\nonumber}
%\newcommand{\refeq}[1]{Eq.~(\ref{eq:#1})}
%\newcommand{\refeqs}[2]{Eqs.~(\ref{eq:#1})-(\ref{eq:#2})}
%\newcommand{\reffig}[1]{Fig.~\ref{fig:#1}}
%\newcommand{\refsec}[1]{Section \ref{sec:#1}}
%\newcommand{\reftab}[1]{Table~\ref{tab:#1}}
%\newcommand{\refapp}[1]{Appendix~\ref{sec:#1}}
%\newcommand{\order}[1]{\mathcal{O}\left({#1}\right)}
%\newcommand{\para}{\parallel}

%\def \Re{\textrm{Re}}
%\def \Im{\textrm{Im}}

%% Physics
%\newcommand{\alphas}{\alpha_\mathrm{s}}
%\newcommand{\alphae}{\alpha_\mathrm{e}}
%\newcommand{G_{F}ermi}{G_\mathrm{F}}
%\newcommand{\GeV}{\,\mathrm{GeV}}
%\newcommand{\MeV}{\,\mathrm{MeV}}
%\newcommand{\amp}[1]{\mathcal{A}\left({#1}\right)}
%\newcommand{\wilson}[2][{}]{\mathcal{C}_{#2}^{\mathrm{#1}}}
%\newcommand{\bra}[1]{\left\langle{#1}\right\vert}
%\newcommand{\ket}[1]{\left\vert{#1}\right\rangle}
%\newcommand{\braket}[1]{\left\langle #1 \right\rangle}

%------------------------
% transversity amplitudes
%------------------------

\def \azeL{{A_0^L}}
\def \azeR{{A_0^R}}
\def \apaL{{A_\|^L}}
\def \apaR{{A_\|^R}}
\def \apeL{{A_\bot^L}}
\def \apeR{{A_\bot^R}}

%--------
% angles
%--------

\def \thl {{\theta_\ell}}
\def \thK {{\theta_{K}}}
\def \barthl {{\bar{\theta}_l}}
\def \barthK {{\bar{\theta}_{K^*}}}


%
%
%--+----1----+----2----+----3----+----4----+----5----+----6----+----7---+----8
\section{The Effective Hamiltonian \label{sec:eff:Ham}}

Rare semileptonic $|\Delta B| = |\Delta S| = 1$ decays
are described by an effective Hamiltonian
\begin{align}
  \label{eq:Heff}
  {\cal{H}}_{\rm eff}= 
   - \frac{4\, G_F}{\sqrt{2}}  V_{tb}^{} V_{ts}^\ast \,\frac{\alpha_e}{4 \pi}\,
     \sum_i \wilson[]{i}(\mu)  \mathcal{O}_i(\mu).
\end{align}
Here, $G_F$ denotes Fermi's constant, $\alpha_e$ the fine structure constant and
unitarity of the Cabibbo-Kobayashi-Maskawa (CKM) matrix $V$ has been used. The
subleading contribution proportional to $V_{ub}^{} V_{us}^\ast$ has been
neglected. \todo{Add up-part.}

The renormalization scale $\mu$, which appears in the short-distance couplings
$\wilson[]{i}$ and the matrix elements of the operators $\mathcal{O}_i$, is of the order
of the $b$-quark mass. In the following we suppress the dependence of the Wilson
coefficients $\wilson{i}$ on the scale $\mu$.

In the SM $b \to s\,\ell^+\ell^-$ processes are mainly governed by the operators
$\mathcal{O}_{7,9,10}$ which will be referred to as the SM operator basis. Beyond the SM
chirality-flipped ones $\mathcal{O}_{7',9',10'}$, collectively denoted here by SM', may
appear. The SM and SM' operators are written as \cite{Bobeth:2007dw,
  Altmannshofer:2008dz, Kruger:2005ep}
\begin{equation}
\begin{aligned}
  \mathcal{O}_{7(7')} & = \frac{m_b}{e}\!\left\lbrack\bar{s} \sigma^{\mu\nu} P_{R(L)} b\right\rbrack F_{\mu\nu} \,,
\\
  \mathcal{O}_{9(9')} & = \left\lbrack\bar{s} \gamma_\mu P_{L(R)} b\right\rbrack\!\left\lbrack\bar{\ell} \gamma^\mu \ell\right\rbrack \,,
\\left\lbrack0.1cm]
  \mathcal{O}_{10(10')} & = \left\lbrack\bar{s} \gamma_\mu P_{L(R)} b\right\rbrack\!\left\lbrack\bar{\ell} \gamma^\mu \gamma_5 \ell\right\rbrack \,.
\end{aligned}
\label{eq:SM:ops}
\end{equation}
Furthermore, we allow for scalar and pseudo-scalar operators, referred to as S
and P,
\begin{equation}
\begin{aligned}
    \mathcal{O}_{S(S')}   & = \left\lbrack\bar{s} P_{R(L)} b\right\rbrack\!\left\lbrack\bar{\ell} \ell\right\rbrack \,,
\\left\lbrack0.1cm]
    \mathcal{O}_{P(P')}   & = \left\lbrack\bar{s} P_{R(L)} b\right\rbrack\!\left\lbrack\bar{\ell} \gamma_5 \ell\right\rbrack \,,
\end{aligned}
\label{eq:psd-scalar:ops}
\end{equation}
which includes the chirality-flipped ones, as well as tensor operators, referred
to as T and T5,
\begin{equation}
\begin{aligned}
  \mathcal{O}_T   & = \left\lbrack\bar{s} \sigma_{\mu\nu} b\right\rbrack\!\left\lbrack\bar{\ell} \sigma^{\mu\nu} \ell\right\rbrack \,,
\\
  \mathcal{O}_{T5} & = \left\lbrack\bar{s} \sigma_{\mu\nu} b\right\rbrack \left\lbrack\bar{\ell} \sigma^{\mu\nu} \gamma_5 \ell\right\rbrack \,.
\end{aligned}
\label{eq:tensor:ops}
\end{equation}
Note that $\mathcal{O}_{T5} = - i/2\, \varepsilon^{\mu\nu\alpha\beta}
    \left\lbrack\bar{s} \sigma_{\mu\nu} b\right\rbrack\!\left\lbrack\bar{\ell} \sigma_{\alpha\beta} \ell\right\rbrack
  = - \mathcal{O}_{TE}/2$,
see Eq.~(\ref{eq:gamma5rel}),
as commonly used in the literature \cite{Bobeth:2007dw,Kim:2007fx, Alok:2010zd}.

%
%
%--+----1----+----2----+----3----+----4----+----5----+----6----+----7---+----8
\section{Angular distribution \label{sec:ang:dist}}

The differential decay rate of $\bar{B}\to\bar{K}^* (\to \bar{K}\pi)\,
\ell^+\ell^-$ can, after summing over the lepton spins, assuming an  on-shell $\bar{K}^*$
of narrow width, and integrating
over the $\bar{K}\pi$-invariant mass, be written as
\begin{equation}
\begin{split}
  \frac{8 \pi}{3} & \frac{d^4 \Gamma}{d q^2\, d\!\cos\thl\, d\!\cos\thK\, d\phi} = 
\\left\lbrack0.1cm]
    & (J_{1s} + J_{2s} \cos\!2\thl + J_{6s} \cos\thl) \sin^2\!\thK
\\left\lbrack0.1cm]
  + & (J_{1c} + J_{2c} \cos\!2\thl + J_{6c} \cos\thl) \cos^2\!\thK  
\\left\lbrack0.2cm]
  + & (J_3 \cos 2\phi + J_9 \sin 2\phi) \sin^2\!\thK \sin^2\!\thl
\\left\lbrack0.2cm] 
  + & (J_4 \cos\phi + J_8  \sin\phi) \sin 2\thK \sin 2\thl 
\\left\lbrack0.2cm]
  + & (J_5 \cos\phi  + J_7 \sin\phi ) \sin 2\thK \sin\thl \, ,
\end{split}
\label{eq:anganal}
\end{equation}
with twelve angular coefficients $J_i=J_i(q^2)$ times the angular
dependence. The angles are defined as $i)$ the angle $\thl$ between $\ell^-$ and
$\bar{B}$ in the $(\ell^+\ell^-)$ center of mass system (cms), $ii)$ the angle
$\thK$ between $K^-$ and the negative direction of flight of the $\bar{B}$ in the 
$(K^-\pi^+)$ cms --- this corresponds to the angle between the $K^-$ and
the $(K^-\pi^+)=K^*$ direction of flight in the $\bar{B}$ rest frame --- and 
$iii)$ the angle $\phi$ between the two decay planes spanned by the 3-momenta
of the $(K^-\pi^+)$- and $(\ell^+\ell^-)$-systems, respectively 
\cite{Bobeth:2008ij, Altmannshofer:2008dz, Kruger:1999xa, Kruger:2005ep}.

Within the (SM+SM') operator basis holds $J_{6c} = 0$. A nonvanishing $J_{6c}$
arises only from interference between the operator sets (SM+SM') and S
\cite{Altmannshofer:2008dz}, (SM+SM') and T, and P and T \cite{Alok:2010zd}.
The explicit expressions of the $J_i$ are given in \ref{sec:ang:obs}.

We denote by 
\begin{align} \label{eq:aveX}
  \braket{J_i} = \int_{q^2_{\rm min}}^{q^2_{\rm max}} dq^2\, J_i(q^2) 
\end{align}
$q^2$-integrated angular observables $J_i$ in bins between $q^2_{\rm min}$ and $q^2_{\rm
  max}$.  For composite observables $X$ we use
  $\braket{X}=X(\braket{J_i})$. We assume in the following that an S-wave background from $\bar K \pi$
around the $K^*(892)$ mass has been removed.

Starting from the $q^2$-integrated decay distribution $d^3\!\braket{\Gamma} /
d\!\cos\thl\, d\!\cos\thK d\phi$ one obtains the integrated decay rate and the
three single-angular differential distributions
\begin{align}
  \label{eq:Gint}
  \braket{\Gamma} & =
    2 \braket{J_{1s}} + \braket{J_{1c}} 
  - \frac{1}{3} \left(2 \braket{J_{2s}} + \braket{J_{2c}} \right)\,, 
\\left\lbrack0.2cm]
  \label{eq:dG:dphi}
  \frac{d\braket{\Gamma}}{d\phi} & =  
      \frac{\braket{\Gamma}}{2\pi} 
    + \frac{2}{3\pi} \braket{J_3} \cos 2\phi 
    + \frac{2}{3\pi} \braket{J_9} \sin 2\phi\,,
\\left\lbrack0.2cm]
  \frac{d\braket{\Gamma}}{d\!\cos\thl} & =
      \braket{J_{1s}} + \frac{\braket{J_{1c}}}{2}
    + \left(\braket{J_{6s}} + \frac{\braket{J_{6c}}}{2} \right) \cos\thl 
\nonumber \\
    & \qquad+ \left(\braket{J_{2s}} + \frac{\braket{J_{2c}}}{2} \right) \cos 2\thl \,,
  \label{eq:dG:dcosthL}
\\left\lbrack0.2cm]
  \frac{d\braket{\Gamma}}{d\!\cos\thK} & = 
  \frac{3}{2} \left\lbrack
       \left(\braket{J_{1s}} - \frac{1}{3} \braket{J_{2s}} \right) \sin^2\!\thK  \right.
\nonumber \\
   & \qquad \left. + \left(\braket{J_{1c}} - \frac{1}{3} \braket{J_{2c}} \right) \cos^2\!\thK
    \right\rbrack
    \label{eq:dG:dcosthK}
\end{align}
after integration over either all or the remaining two angles, respectively. 

%
The lepton forward-backward asymmetry $A_{\rm FB}$ can be written as 
\begin{align} \label{eq:afb}
  \braket{A_{\rm FB}} \braket{\Gamma} & 
   = \braket{J_{6s}} + \frac{\braket{J_{6c}}}{2} \, ,
\end{align}
see Eq.~(\ref{eq:dG:dcosthL}). 
%
The extraction of $J_{4,5,7,8}$ has been discussed in \cite{Bobeth:2008ij}.  For
alternative methods to obtain the $J_i$, see for example \cite{Kim:2007fx,
  Altmannshofer:2008dz,Matias:2012qz}.

The longitudinal
$K^*$ polarization fraction $F_L$ can model-independently be defined  as
\begin{align}
  \label{eq:dGdcK}
  \frac{1}{\braket{\Gamma}} \frac{d\!\braket{\Gamma}}{d\!\cos\thK} & =
     \frac{3}{4} \braket{F_T} \sin^2\!\thK
   + \frac{3}{2} \braket{F_L} \cos^2\!\thK \,.
   \end{align}
{}From comparison with Eq.~(\ref{eq:dG:dcosthK}) one can read off
\begin{align}
  \label{eq:defFL}
  \braket{F_L} & = \frac{1}{\braket{\Gamma}} \left(
     \braket{J_{1c}}  - \frac{1}{3}\, \braket{J_{2c}} \right)\,, \\left\lbrack0.4cm]
  \label{eq:defFT}
  \braket{F_T} & = \frac{2}{\braket{\Gamma}} \left(
     \braket{J_{1s}} - \frac{1}{3} \braket{J_{2s}} \right)\,,
\end{align}
where $F_T + F_L = 1$.

In the experimental analyses by the collaborations
Belle \cite{:2009zv}, BaBar \cite{:2012vwa}, CDF \cite{BKll:CDF:ICHEP:2012}
and LHCb \cite{ Aaij:2011aa} 
the distribution
\begin{align}
  \label{eq:dGdcL}
  \frac{1}{\braket{\Gamma}} \frac{d\!\braket{\Gamma}}{d\!\cos\thl} & = 
      \frac{3}{4} \braket{F_L} (1 - \cos^2\!\thl)  
\\ \nonumber
&   + \frac{3}{8} \braket{F_T} (1 + \cos^2\!\thl) 
   + \braket{A_{\rm FB}} \cos\thl \, &
\end{align}
is at least partially  employed.
We stress that the latter is based on 
[\cf Eqs.~(\ref{eq:J1s}) - (\ref{eq:J2c})]
\begin{align}
  \label{eq:J1:mlzero}
  J_{1s} & = 3\, J_{2s} \, , & 
  J_{1c} & = -J_{2c} \, ,
\end{align}
which is broken by $m_\ell \neq 0$ and/or in the presence of S, P, T or T5
contributions. Therefore, results for $F_L$  based on Eq.~(\ref{eq:dGdcL})
do not hold in full generality.

Note that in cases where Eq.~(\ref{eq:J1:mlzero}) holds, such as the SM with
lepton masses neglected, $F_L = (|A_0^L|^2 + |A_0^R|^2)/\Gamma= -
J_{2c}/\Gamma$. Furthermore, $\braket{J_{2s}} = 3/16\, \braket{\Gamma} (1 -
\braket{F_L})$ and $\braket{J_{2c}} = - 3/4\, \braket{\Gamma} \braket{F_L}$.

%
%
%--+----1----+----2----+----3----+----4----+----5----+----6----+----7---+----8
\section{Angular observables \label{sec:ang:obs}}

The $J_i(q^2)$ of Eq.~(\ref{eq:anganal}) can be conveniently expressed within
the (SM+SM') operator basis with the help of seven transversity amplitudes,
$A_{0,\perp,\parallel}^{L,R}$ and $A_t$, \cite{Kruger:2005ep}. The operators S
require an additional amplitude $A_S$, whereas the set P can be absorbed into
the amplitude $A_t$ \cite{Altmannshofer:2008dz}. In the presence of tensor
operators T and T5, six additional transversity amplitudes $A_{ij}$ need to be
introduced, with $ij = \lbrace\parallel\perp,\, t0,\, t\!\perp,\, t\!\parallel,\,
0\!\perp,\, 0\!\parallel \rbrace$, see \refsec{calculation}. In the
complete basis (SM+SM') $+$ (S+P) $+$ (T+T5) we obtain
\begin{align}
  \label{eq:J1s}
  \frac{4}{3} J_{1s} & = 
    \frac{(2 + \beta_\ell^2)}{4} \left\lbrack|\apeL|^2 + |\apaL|^2 + (L\to R) \right\rbrack
    + \frac{4 m_\ell^2}{q^2} \Re\left(\apeL^{}\apeR^* + \apaL^{}\apaR^*\right)
\\ \nonumber & \hspace{0.5cm}
  + 4\, \beta_\ell^2 \big(|A_{0\perp}|^2 + |A_{0\parallel}|^2\big)
       + 4\, (4- 3 \beta_\ell^2)\, \big(|A_{t\perp}|^2 + |A_{t\parallel}|^2\big)
\\ \nonumber & \hspace{0.5cm}
  + 8\sqrt{2} \frac{m_\ell}{\sqrt{q^2}} \Re\left\lbrack
      (\apaL + \apaR) A_{t\parallel}^* {+ (\apeL + \apeR) A_{t\perp}^*}
    \right\rbrack , 
\\left\lbrack0.2cm]
  \frac{4}{3} J_{1c} & = 
    |\azeL|^2 +|\azeR|^2
  + \frac{4 m_\ell^2}{q^2} \Big[|A_t|^2 + 2\,\Re(\azeL^{}\azeR^*) \Big]
  + \beta_\ell^2 |A_S|^2 
\\ \nonumber & \hspace{0.5cm}
  + 8\, (2 - \beta_\ell^2) |A_{t0}|^2 + 8\, \beta_\ell^2 |A_{\parallel\perp}|^2
  + 16\, \frac{m_\ell}{\sqrt{q^2}} \Re\Big[ (\azeL + \azeR) A_{t0}^*\Big],
\\left\lbrack0.2cm]
  \frac{4}{3} J_{2s} & =
    \frac{\beta_\ell^2}{4} \bigg[ |\apeL|^2+ |\apaL|^2 + (L\to R)
       - 16 \, \big(|A_{t\perp}|^2 + |A_{t\parallel}|^2 + |A_{0\perp}|^2 + |A_{0\parallel}|^2 \big) \bigg],
\\left\lbrack0.2cm]
  \label{eq:J2c}
  \frac{4}{3} J_{2c} & =
    -\beta_\ell^2 \bigg[|\azeL|^2 + |\azeR|^2
        - 8 \, \big(|A_{t0}|^2 + |A_{\parallel\perp}|^2 \big)\bigg],
\\left\lbrack0.2cm]
  \frac{4}{3} J_3 & =
    \frac{\beta_\ell^2}{2} \bigg[ |\apeL|^2 - |\apaL|^2  + (L\to R)
      + 16\, \big( |A_{t\parallel}^{}|^2 - |A_{t\perp}^{}|^2 +
      |A_{0\parallel}^{}|^2 - |A_{0\perp}^{}|^2 \big) \bigg],
\\left\lbrack0.2cm]
  \frac{4}{3} J_4 & =
    \frac{\beta_\ell^2}{\sqrt{2}} \Re \bigg[\azeL^{}\apaL^* + (L\to R) 
    - 8 \, \sqrt{2}\, \Big(A_{t0}^{} A_{t\parallel}^*
                         + A_{\parallel\perp}^{} A_{0\parallel}^* \Big) \bigg],
\\left\lbrack0.2cm]
  \frac{4}{3} J_5 & =
    \sqrt{2}\beta_\ell\, \Re \bigg[\azeL^{}\apeL^* - (L\to R)
      - 2\, \sqrt{2} A_{t\parallel}^{}  A_S^*
      - \frac{m_\ell}{\sqrt{q^2}} \Big( [\apaL + \apaR] A_S^* 
\\ \nonumber & \hspace{3.7cm}
      + 4\, \sqrt{2}\, A_{0\parallel}^{} A_t^{*} 
      {
      - 4\, \sqrt{2}\, [\azeL - \azeR]^{} A_{t\perp}^*}
      - 4\, [\apeL - \apeR] A_{t0}^* \Big) \bigg],
\\left\lbrack0.2cm]
  \frac{4}{3} J_{6s} & =
    2\,\beta_\ell\, \Re \bigg[\apaL^{}\apeL^* - (L\to R)
    {
    + 4\, \sqrt{2}\, \frac{ m_\ell}{\sqrt{q^2}} \Big(
[\apeL - \apeR] A_{t\parallel}^* + [\apaL - \apaR] A_{t\perp}^*}
   \Big) \bigg],
\\left\lbrack0.2cm]
  \label{eq:j6c}
  \frac{4}{3} J_{6c} & =
    4\, \beta_\ell\, \Re \bigg[ 2\, A_{t0}^{} A_S^* +
      \frac{m_\ell}{\sqrt{q^2}} \big[(\azeL + \azeR) A_S^* 
           + 4\, A_{\parallel\perp}^{} A_t^* \big] \bigg],
\\left\lbrack0.2cm]
  \frac{4}{3} J_7 & = \label{eq:J7}
    \sqrt{2} \beta_\ell\, \Im \bigg[ \azeL^{}\apaL^* - (L\to R)
      {
      + 2\, \sqrt{2} A_{t\perp}^{}  A_S^*}
      + \frac{m_\ell}{\sqrt{q^2}} \Big( [\apeL + \apeR] A_S^* 
\\ \nonumber & \hspace{3.7cm}
      + 4\, \sqrt{2}\, A_{0\perp}^{} A_t^{*} 
      + 4\, \sqrt{2}\, [\azeL - \azeR]^{} A_{t\parallel}^{*}
      - 4\, [\apaL - \apaR] A_{t0}^* \Big)\bigg],
\\left\lbrack0.2cm]
  \frac{4}{3} J_8 & = 
    \frac{\beta_\ell^2}{\sqrt{2}}\, \Im \bigg[
     \azeL^{}\apeL^* + (L\to R) \big) \bigg],
\\left\lbrack0.2cm]
  \frac{4}{3} J_9 & =
    \beta_\ell^2\, \Im \bigg[\apeL \apaL^{*} + (L\to R) \big) \bigg]\,,
\end{align}
where the lepton mass $m_\ell$ has been kept and 
\begin{align}
 \beta_\ell & = \sqrt{1 - \frac{4\, m_\ell^2}{q^2}}.
\end{align}

%
%
%--+----1----+----2----+----3----+----4----+----5----+----6----+----7---+----8
\section{Optimised observables \label{sec:opt:obs}}
  
%
%
%--+----1----+----2----+----3----+----4----+----5----+----6----+----7---+----8
\section{Transversity amplitudes \label{sec:trAmps}}

The following transversity amplitudes receive only contributions from the operators
in Eqs.~(\ref{eq:SM:ops}) -- (\ref{eq:tensor:ops}) which are factorizable (at
lowest order in QED)
\begin{align}
  \label{eq:trAmp:At}
  A_t & = N \frac{\sqrt{\lambda}}{\sqrt{q^2}}
    \left\lbrack 2\, (\wilson[]{10} - \wilson[]{10'}) + 
     \frac{q^2}{m_\ell} \frac{(\wilson[]{P} - \wilson[]{P'})}{(m_b + m_s)}\right\rbrack A_0,
\\
  \label{eq:trAmp:As}
  A_S & = -2 N \sqrt{\lambda}\, \frac{(\wilson[]{S} - \wilson[]{S'})}{(m_b + m_s)} A_0, 
\\
  \label{eq:tensorAmpFirst}
  A_{\parallel\!\perp\,(t0)} & = \pm N \frac{\wilson[]{T(5)}}{M_{K^*}}\, \Big[ 
    (M_B^2 + 3\, M_{K^*}^2 - q^2)\, T_2\\
    & \quad - \frac{\lambda}{M_B^2 - M_{K^*}^2}\, T_3
     \Big], \nonumber
\\left\lbrack0.2cm]
  A_{t\perp\,(0\perp)} & 
    = \pm 2 N \, \frac{\sqrt{\lambda}}{\sqrt{q^2}}\,\wilson[]{T(5)}\, T_1, &
\\left\lbrack0.2cm]
  \label{eq:tensorAmpLast}
  A_{0\parallel\,(t\parallel)} &
    = \pm 2 N \, \frac{(M_B^2 - M_{K^*}^2)}{\sqrt{q^2}}\, \wilson[]{T(5)}\, T_2. &
\end{align}
The upper and lower sign in \refeqs{tensorAmpFirst}{tensorAmpLast} refers 
$C_T$ and $C_{T5}$, respectively. The normalization factor $N$ is given as
\begin{align}
  \label{eq:trAmp:norm:factor}
  N & = G_F\, \alpha_e\, V_{tb}^{}V_{ts}^{*}\,
    \sqrt{\frac{q^2 \, \beta_\ell \,\sqrt{\lambda}}{3 \cdot 2^{10}\, \pi^5\, M_B^3}}
\end{align}
and the $B \to K^*$ form factors $V$, $A_{0,1,2}$, $T_{1,2,3}$ are defined as in
\cite{Beneke:2001at, Bobeth:2010wg, Ball:2004rg, Altmannshofer:2008dz,
  Kruger:2005ep, Alok:2010zd}. Above we use the K{\"a}ll{\'e}n-function 
$\lambda= \lambda(m_B^2, m_{K^*}^2, q^2)$
\begin{align}
  \label{eq:defLambda}
  \lambda(a, b, c)
    & = a^2 + b^2 + c^2 - 2(ab + ac + bc)\,.
\end{align}

In the remaining transversity amplitudes, the contributions from the hadronic part
of the effective $|\Delta B|=1$ Hamiltonian $\propto \mathcal{O}_{i\leq 6,8}$ are taken
into account in various degrees of sophistication, depending also on the $q^2$-region.
In the following we have 
\begin{enumerate}
\item naive factorization,
\item low-$q^2$: QCDF with/without form-factor relations,
\item low-$q^2$: Khodjamirian et al.,
\item high-$q^2$: local OPE,
\item additional contributions parameterizing subleading contributions,
\end{enumerate}
which will be given in the sub-sequent sections.

%
%--+----1----+----2----+----3----+----4----+----5----+----6----+----7---+----8
\subsection{Naive factorization \label{sec:trAmps:naive}}

Here the transversity amplitudes contain the contributions from the operators
in Eqs.~(\ref{eq:SM:ops}) -- (\ref{eq:tensor:ops}) which are factorizable.
Within naive factorization the transversity  amplitudes read
\begin{align}
  A_\perp^{L,R} & = \sqrt{2} N \sqrt{\lambda}
    \Big\lbrace\left\lbrack\left(\wilson{9} + \wilson[]{9'}\right) \mp \left(\wilson{10} + \wilson[]{10'}\right)\right\rbrack\frac{V}{M_B + M_{K^*}}
      + \frac{2 m_b}{q^2}\left(\wilson{7} + \wilson[]{7'}\right)T_1\Big\rbrace,
\\left\lbrack0.2cm]
  A_\parallel^{L,R} & = -N \sqrt{2}(M_B^2 - M_{K^*}^2) \times \Big\lbrace
\\ 
\nonumber & \hskip 2.5cm  
    \left\lbrack\left(\wilson{9} - \wilson[]{9'}\right) \mp \left(\wilson{10} - \wilson[]{10'}\right)\right\rbrack\frac{A_1}{M_B - M_{K^*}}
      + \frac{2 m_b}{q^2}\left(\wilson{7} - \wilson[]{7'}\right)T_2 \Big\rbrace,
\\left\lbrack0.2cm]
  A_0^{L,R} & = -\frac{N}{2 M_{K^*}\sqrt{q^2}} \times \Big\lbrace
\\
\nonumber & \hskip 0.5cm
    \Big[ \left(\wilson{9} - \wilson[]{9'}\right) \mp \left(\wilson{10} - \wilson[]{10'}\right) \Big]
    \Big[(M_B^2 - M_{K^*}^2 - q^2)(M_B + M_{K^*}) A_1 - \frac{\lambda}{M_B + M_{K^*}} A_2 \Big]
\\ \nonumber
    & \hskip 0.8cm + 2 m_b \left(\wilson{7} - \wilson[]{7'} \right)
        \Big[\left(M_B^2 + 3 M_{K^*}^2 - q^2\right)T_2 - \frac{\lambda}{M_B^2 - M_{K^*}^2}T_3\Big]
      \Big\rbrace  ,
\end{align}
Non-factorizable contributions from $\mathcal{O}_{i\leq 6}$  are taken into account
by using effective Wilson coefficients $\wilson[]{7,8,9} \to \wilson[eff]{7,8,9}$
and similarly for the corresponding primed Wilson coefficients, when assuming
also primed current-current and QCD-penguin operators $\mathcal{O}_{i\leq 6'}$ to have
non-vanishing Wilson coefficients.
The effective Wilson coefficients contain usually only the 1-loop matrix elements 
of the 4-quark operators at the quark level, which cancel the leading $\mu_b$
dependence in $\wilson[]{9}$, but not so for $\wilson[]{7,8}$.

%
%--+----1----+----2----+----3----+----4----+----5----+----6----+----7---+----8
\subsection{\boldmath Low $q^2$ QCDF \label{sec:trAmps:QCDF}}

%
%--+----1----+----2----+----3----+----4----+----5----+----6----+----7---+----8
\subsection{\boldmath Low $q^2$ Khodjamirian \label{sec:trAmps:QCDF}}

%
%--+----1----+----2----+----3----+----4----+----5----+----6----+----7---+----8
\subsection{\boldmath High $q^2$ local OPE \label{sec:trAmps:QCDF}}

%
%--+----1----+----2----+----3----+----4----+----5----+----6----+----7---+----8
\subsection{\boldmath Subleading parameterizations \label{sec:trAmps:QCDF}}

%
%
%--------+---------+---------+---------+---------+---------+---------+---------+
\section{$\bar{B}\to\bar{K}^*(\to \bar{K}\pi)\,\ell^+\ell^-$ Matrix Element 
  \label{sec:calculation}}

We present here the parametrization of the hadronic matrix element used to
calculate the decay $\bar{B}\to\bar{K}^*(\to \bar{K}\pi)\,\ell^+\ell^-$,
\begin{multline}
  \label{eq:BKpill:ME:X}
  {\cal M} =
  {\cal F} \Big( X_S \left\lbrack\bar{\ell} \ell\right\rbrack
    + X_P \left\lbrack\bar{\ell} \gamma_5 \ell\right\rbrack
    + X^{\mu}_V \left\lbrack\bar{\ell} \gamma_{\mu} \ell\right\rbrack\\
    + X^{\mu}_A \left\lbrack\bar{\ell} \gamma_{\mu} \gamma_5 \ell\right\rbrack
    + X^{\mu\nu}_T \left\lbrack\bar{\ell} \sigma_{\mu\nu} \ell\right\rbrack
    \Big)\,.
\end{multline}
We define
\begin{align}
  {\cal F} & = i \frac{G_{F} \alpha_e}{\sqrt{2} \pi}\, V_{tb}^{} V_{ts}^{\ast}\,\,
               g_{K^*K\pi} D_V\, 2 |\vec{p}_K|\,,\\
\intertext{and use $\vec{p}_K$, the three momentum of the $\bar{K}$ in the $\bar{K}\pi$ cms,
}
  |\vec{p}_K| & = \frac{\sqrt{\lambda\left(M^2_{K^*}, M^2_K, M^2_\pi\right)}}{2\, M_{K^*}}\,.
\end{align}

Using this parametrization, we obtain the hadronic tensors
\begin{gather}
    X_S = -\frac{i}{4 N} \cos\thK\, A_S\,,\\
    X_P = +\frac{i}{2 N} \cos\thK\, \frac{m_\ell}{\sqrt{q^2}}\, A_{t}\,,
\end{gather}
\begin{align}
\label{eq:XVA}
  X^{\mu}_{V,A} & = \frac{i}{4 N} \cos\thK\, \varepsilon^\mu(0) \, (A_{0}^R \pm A_{0}^L)
\\
\nonumber
  & + \frac{i}{8 N} \sin\thK\\
\nonumber
  & \times \Big(\varepsilon^\mu(+)\, e^{+i \phi}\!\left\lbrack(A_\parallel^R + A_\perp^R) \pm (A_\parallel^L + A_\perp^L) \right\rbrack\\
\nonumber
  & +
  \varepsilon^\mu(-)\, e^{-i \phi}\!\left\lbrack(A_\parallel^R - A_\perp^R) \pm (A_\parallel^L - A_\perp^L) \right\rbrack\Big),
\end{align}
\begin{align}
    \raisetag{4.4ex}
  X^{\mu\nu}_{T} & = \frac{\cos\thK}{N}
  \left(\varepsilon^\mu(t)\,\varepsilon^\nu(0) A_{t0} - \varepsilon^\mu(+)\,\varepsilon^\nu(-)A_{\parallel\perp}\right)
\\
\nonumber
  & + \frac{\sin\thK}{\sqrt{2} N}\, \varepsilon^\mu(t)\\
\nonumber
  & \times \left(\varepsilon^\nu(+)\, e^{ i\phi}\, [A_{t\parallel}\, {+ A_{t\perp}}] + 
    \varepsilon^\nu(-)\, e^{-i\phi}\, [A_{t\parallel} {- A_{t\perp}}]\right)\\
\nonumber
  & - \frac{\sin\thK}{\sqrt{2} N}\, \varepsilon^\mu(0)\\
\nonumber
  & \times \left(\varepsilon^\nu(+)\, e^{i\phi} [A_{0\perp} + A_{0\parallel}] + 
    \varepsilon^\nu(-)\, e^{-i\phi}[A_{0\perp} - A_{0\parallel}]\right),
\end{align}
where the polarization vectors $\varepsilon^\mu(n)$ in the $\bar{B}$ meson rest
frame read \cite{Altmannshofer:2008dz}
\begin{equation}
\label{eq:pol-vectors:eps}
\begin{aligned}
  \varepsilon^\mu(\pm) & = \frac{1}{\sqrt{2}}(0,1,\mp i,0)\,,\\
  \varepsilon^\mu(0)   & = \frac{1}{\sqrt{q^2}}(-q_z, 0, 0, -q_0)\,,\\
  \varepsilon^\mu(t)   & = \frac{1}{\sqrt{q^2}}(q_0, 0, 0, q_z) \, .
\end{aligned}
\end{equation}
We choose the $z$-axis in this frame along the $\bar{K}^*$ direction of flight
and $q_0$ ($q_z$) denotes the timelike (spacelike) component of the four
momentum $q^\mu$. The polarization vectors fulfill the completeness relations
\begin{equation}
\begin{aligned}
    g_{nn'} & = \varepsilon^\dagger_\mu(n)\, \varepsilon^\mu(n')\,,\\
 g_{\mu\nu} & = \sum_{n,n'} \varepsilon^\dagger_\mu(n)\, \varepsilon_\nu(n')\, g_{nn'}
\end{aligned}\label{eq:heldecomp}
\end{equation}
with $g_{nn'} = \mbox{diag}(+,-,-,-)$ for $n,n' = t, \pm, 0$. We use
the relation \refeq{heldecomp}  to insert the full set of polarization vectors
$\varepsilon^\mu(n)$ between the hadronic and leptonic currents, and introduce
the helicity amplitudes $H_{a n_1 \dots n_l}$ for arbitrary Dirac
structures $\Gamma^{\mu_1\dots\mu_l}$,
\begin{gather}
    \langle \bar{K}^*(k, \eta(a)) |\bar{s}\Gamma_{\mu_1\dots\mu_l} b|\bar{B}(p)\rangle\\
    \nonumber
    = \sum_{n_i,n'_i}
          \!\langle \bar{K}^*(k,\eta(a))|\bar{s}\Gamma^{\nu_1\dots\nu_l}b|\bar{B}(p)\rangle
      \prod_{i=1}^l \varepsilon^{\dagger}_{\nu_i}\!(n_i) g^{n_i n'_i} \varepsilon_{\mu_i}\!(n'_i)\\
    \equiv \sum_{n_i}
    H^\Gamma_{a n_1\dots n_l} \prod_{i=1}^l g^{n_i n_i} \varepsilon_{\mu_i}\!(n_i)\,.
\end{gather}
The tensorial transversity amplitudes $A_{ij}$ are related to the helicity
amplitudes $H_{a n_1 n_2}$ by means of
\begin{align}
    A^\Gamma_{0\perp}     & = \frac{1}{2} \big(H^\Gamma_{+0+} + H^\Gamma_{-0-}\big) &
    A^\Gamma_{0\para}     & = \frac{1}{2} \big(H^\Gamma_{+0+} - H^\Gamma_{-0-}\big)\nn\\
    A^\Gamma_{t\perp}     & = \frac{1}{2} \big(H^\Gamma_{+t+} - H^\Gamma_{-t-}\big) &
    A^\Gamma_{t\para}     & = \frac{1}{2} \big(H^\Gamma_{+t+} + H^\Gamma_{-t-}\big)\nn\\
    A^\Gamma_{\para\perp} & = H^\Gamma_{0+-} &
    A^\Gamma_{t0}         & = H^\Gamma_{0t0}
\end{align}
and
\begin{equation}
    A_{ij} = 2\, N \sum_{\Gamma=T,T5} C_\Gamma  A^\Gamma_{ij}\,.
\end{equation}
Note that the factor 2 above emerges from the relation $H_{a ij}^\Gamma =
 - H_{a ji}^\Gamma$, which is due to the asymmetry of $\sigma^{\mu\nu}$ under
 $\mu\leftrightarrow \nu$.
The polarization vectors of the $\bar{K}^*$ for polarizations $a = \pm,0$ in the $\bar B$ cms read
\begin{equation}
\label{eq:pol-vectors:eta}
\begin{aligned}
  \eta^\mu(\pm) & = \frac{1}{\sqrt{2}}(0,1,\pm i,0)\,,\\
  \eta^\mu(0)   & = \frac{1}{M_{K^*}}(-q_z, 0, 0, M_B-q_0) \, .
\end{aligned}
\end{equation}
This approach generalizes the concept of the transversity amplitudes,
cf. e.g. Refs.  \cite{Faessler:2002ut,Kruger:2005ep,Altmannshofer:2008dz}, to
which we also refer for the definition of the remaining transversity amplitudes
$A_{i}$, $i=0,\perp,\parallel,t,S$.

We employ $\gamma_5 = i/(4\,!)\, \varepsilon_{\alpha\beta\mu\nu} \gamma^\alpha
\gamma^\beta \gamma^\mu \gamma^\nu$, such that
\begin{equation}
\label{eq:gamma5rel}
\begin{aligned}
  \mbox{Tr}[\gamma^\alpha \gamma^\beta \gamma^\mu \gamma^\nu \gamma_5] & =
     4\, i\, \varepsilon^{\alpha\beta\mu\nu}\, ,\\
  \sigma^{\alpha\beta} \gamma_5 & = 
     -\frac{i}{2}\,\varepsilon^{\alpha\beta\mu\nu} \sigma_{\mu\nu}
\end{aligned}
\end{equation}
with $\sigma_{\mu\nu} = i/2\, [\gamma_\mu,\,
\gamma_\nu]$, and $\varepsilon_{0123} = -\varepsilon^{0123} = 1$.


\chapter{$b$-Baryon decays}

\section{Hadronic Matrix Elements}
\label{physics:b-mesons:hme}

\subsection{$1/2^+\to 1/2^+$ Form Factors}
\label{physics:b-mesons:hme:1/2+to1/2+}

For the transition of $J^P = 1/2^+$ to $J'^P = 1/2^+$ state, we use the same parametrization
of the matrix elements as in \cite{Boer:2014kda}. For the vector current, we have
\begin{equation}
\label{eq:physics:b-baryons:1/2+to1/2+:hel-ff-V}
\begin{aligned}
    % V, time-like
    \eps_\mu^*(t) \, \bra{\Lambda(k, s_\Lambda)} \bar{s} \gamma^\mu  b \ket{\Lambda_b(p, s_{\Lambda_b})}
    & \equiv f_t^V(q^2) \, \frac{m_{\Lambda_b} - m_\Lambda}{\sqrt{q^2}}
             \left[ \bar u(k, s_\Lambda) \, u(p,s_{\Lambda_b}) \right]\,,\\
    % V, longitudinal
    \eps_\mu^*(0) \, \bra{\Lambda(k, s_\Lambda)} \bar{s} \gamma^\mu b \ket{\Lambda_b(p, s_{\Lambda_b})}
    & \equiv 2 \, f^{V}_0(q^2) \, \frac{m_{\Lambda_b} + m_\Lambda}{s_+}
             \, (k \cdot\eps^*(0)) \left[ \bar u(k,s_\Lambda) \, u(p,s_{\Lambda_b}) \right]\,,\\
    % V, perpendicular
    \eps_\mu^*(\pm) \, \bra{\Lambda(k, s_\Lambda)} \bar{s} \gamma^\mu b \ket{\Lambda_b(p, s_{\Lambda_b})}
    & \equiv f^{V}_\perp(q^2) \left[ \bar u(k,s_\Lambda) \, \slashed{\eps}^*(\pm) \, u(p,s_{\Lambda_b}) \right]\,,
\end{aligned}
\end{equation}
In the above, $u(p, s)$ represent a Dirac spinor with four-momentum $p$ and spin projection $s$;
$q \equiv p - k$ is the momentum transfer; and $\eps$ is a virtual polarization vector that fulfills
\begin{equation*}
\begin{aligned}
    \eps(t) \cdot q       & = \sqrt{q^2}\,, &
    \eps(\lambda) \cdot q & = 0 \quad\text{for }\lambda=\pm, 0\,.
\end{aligned}
\end{equation*}
For the axialvector current, we use
\begin{equation}
\label{eq:physics:b-baryons:1/2+to1/2+:hel-ff-A}
\begin{aligned}
    % A, time-like
    \eps_\mu^*(t) \, \bra{\Lambda(k, s_\Lambda)} \bar{s} \gamma^\mu \gamma_5 b \ket{\Lambda_b(p, s_{\Lambda_b})}
    & \equiv -f_t^A(q^2) \, \frac{m_{\Lambda_b} + m_\Lambda}{\sqrt{q^2}}
             \left[ \bar u(k, s_\Lambda) \gamma_5 u(p,s_{\Lambda_b}) \right]\,,\\
    % V, longitudinal
    \eps_\mu^*(0) \, \bra{\Lambda(k, s_\Lambda)} \bar{s} \gamma^\mu b \ket{\Lambda_b(p, s_{\Lambda_b})}
    & \equiv 2 \, f^{V}_0(q^2) \, \frac{m_{\Lambda_b} + m_\Lambda}{s_+}
             \, (k \cdot\eps^*(0)) \left[ \bar u(k,s_\Lambda) \, u(p,s_{\Lambda_b}) \right]\,,\\
    % V, perpendicular
    \eps_\mu^*(\pm) \, \bra{\Lambda(k, s_\Lambda)} \bar{s} \gamma^\mu b \ket{\Lambda_b(p, s_{\Lambda_b})}
    & \equiv f^{V}_\perp(q^2) \left[ \bar u(k,s_\Lambda) \, \slashed{\eps}^*(\pm) \, u(p,s_{\Lambda_b}) \right]\,,
\end{aligned}
\end{equation}


\appendix

\chapter{Conventions}

\input{conventions.tex}

\backmatter

%% Glossary
\printglossaries

%% Bibliography
\printbibliography

\end{document}
